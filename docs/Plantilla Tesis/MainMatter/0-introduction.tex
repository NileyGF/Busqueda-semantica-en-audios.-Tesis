%===================================================================================
% Chapter: Introduction
%===================================================================================
\chapter*{Introducción}\label{chapter:introduction}
\addcontentsline{toc}{chapter}{Introducción}
%===================================================================================
El extenso y constante crecimiento de contenido generado en la red en los años recientes, ha 
introducido una apremiante necesidad de buscar en bases de datos cada vez más grandes de multimedia.
Sorprendentemente, mientras existen motores de búsqueda para diversos contenidos como páginas webs, 
imágenes y videos, la música no es accesible de la misma forma.\\ %% TODO rephrase

Los sistemas de recuperación de información (SRI) consisten en tecnologías y métodos diseñados para 
la búsqueda, almacenamiento, recuperación y organización de información. Estos sistemas son esenciales en la 
gestión de grandes cantidades de datos en diversos entornos, como bibliotecas digitales, bases de datos en 
línea y motores de búsqueda en la web.\\ %% TODO I don't like it very much but it is a work in progress version.
Durante las últimas décadas la tecnología computacional se ha desarrollado al punto de 
estar presente, de una manera u otra, en casi todos los procesos cotidianos de los seres humanos.
La velocidad a la que se generan grandes volúmenes de datos en la actualidad supera las capacidades 
computacionales para procesarlos.\\
Una característica fundamental de estos volúmenes de datos es la gran variedad que presentan. Además 
alrededor del 80$\%$ corresponden a datos no estructurados. Y precisamente, la recuperación de información 
se define como; encontrar resultados de naturaleza no estructurada que satisfaga una necesidad de información 
dentro de una gran colección de datos\cite{manning2008introductiontoIR}.\\
Los orígenes de los sistemas de recuperación de información se encuentran intrínsecamente ligados a la expansión de la 
informática y la necesidad de gestionar grandes volúmenes de datos. A medida que la informática evolucionó desde sus 
inicios, el almacenamiento y la recuperación de información se convirtieron en áreas de interés cada vez mayor. 
Surge así la necesidad de desarrollar métodos y sistemas que permitieran a los usuarios acceder de manera 
eficiente a la creciente cantidad de información almacenada.\\ %% TODO Also needs reworking and rephrasing
La recuperación de información utilizando computadoras comienza en la década de 1950, y desde entonces se han 
desarrollado grandes ideas en el campo como: realizar un ranking de los documentos, representación 
vectorial de los documentos y las consultas, agrupamiento de documentos similares, asociación de términos 
con similitudes semánticas, la idea de la frecuencia inversa en documentos y los modelos de semántica latente.\\
% El campo de la recuperación de información continúa cambiando a medida que la computación evoluciona.

En menos de diez años el campo de las redes neuronales ha avanzado a pasos gigantes, principalmente gracias al 
aumento del poder computacional que fue introducido con las GPU. Entre los sectores que se han visto beneficiados por 
esta revolución está la recuperación de información. \\ %% TODO linkeara mejor este párrafo y el siguiente y profundizar en el siguiente
Los modelos de embeddings, como Word2Vec, GloVe y los más recientes, basados en transformers, como BERT y GPT, %% TODO cite all the embeddings models
han demostrado una capacidad excepcional para capturar significado semántico y relación contextual entre palabras 
y frases. Al aplicar estos embeddings a la representación de documentos y consultas, se ha logrado una mejora 
significativa en la precisión y la relevancia de los resultados de recuperación de información.\\ %% TODO está afirmación debería llevar una cita
%% Breve presentación de la problemática
A pesar de que los SRI se han centrado usualmente en recuperar información con forma textual, se ha evidenciado 
la necesidad de adaptar el entorno para recuperar datos de todo tipo (como imágenes, videos, audios). Muchos de 
los acercamientos a resolver ese tipo de problema han sido a través de una representación de datos de tipo multimedia 
como un texto, utilizando los metadatos. \\%% TODO limitaciones de la recuperación utilizand metadatos fijos, o mejor; llevarlo a los problemas de la recuperación tradicional
%% TODO Ahora hablar de que en los últimos años se ha trabajado en recuperación semántica en imágenes (tengo unas cuantas citas para eso).
%% TODO Sin embargo en la música solo hay pocos trabajos en los últimos 10 años. 
%% TODO going foward, presentar Music Information Retrieval como campo y hablar de los content-based retrieval (otra canción o query-by-humming)
%% TODO + sectores poco estudiados => introducir query-by-text retrieval
%% TODO Por qué es importante query-by-text en música, y hablar de que intentaremos hacerlo mejor que solo usar metadatos fijos (entonces las opciones son hacer un embedding del audio directo, o hacer que los metadatos no sean fijos)
Permitir a los usuarios buscar música utilizando oraciones en lenguaje natural introduce una capa adicional de 
complejidad, ya que requiere el puente entre la semántica del lenguaje y la representación de la música en sí misma.
%% Actualidad, novedad científica e importancia teórica y práctica
Uno de los desafíos clave, en la integración de NLP y análisis de audios, se encuentra en la representación 
efectiva de la música en función de las consultas textuales. La naturaleza subjetiva de la música y la 
diversidad en la forma en que los usuarios describen la música pueden dificultar la captura precisa de la intención del usuario. \\
La integración de NLP con sistemas de recuperación de información musical tiene el potencial de mejorar 
la experiencia del usuario al buscar música de manera más natural y accesible.\\ 
%%  Diseño teórico:
%% - problema ciéntifico
%% - objeto de estudio
%% - objetivo general
%% - objetivos espeíficos para cumplir el objetivo general
%% - campo de acción
%% - pregunta ciéntifica
%% - estructura del trabajo




% El Problema de Enrutamiento de Vehículos (en lo adelante, VRP por sus siglas en inglés) fue introducido por primera vez en 1959 en el artículo de George Dantzing y John Ramser \cite{dantzig@vrp}. En este, se hacía referencia a un problema real: la entrega de combustible a las gasolineras. En los años que sucedieron a esta primera aproximación matemática, se han encontrado numerosos enfoques de solución a este problema, tanto con métodos exactos como con heurísticas.

% El objetivo de los VRP es resolver la distribución de mercancías entre un conjunto de almacenes y de clientes mediante la utilización de una flota de vehículos. En la formulación más sencilla, conocida como problema de enrutamiento de vehículos con restricciones de capacidad (CVRP por sus siglas en inglés), se trata de encontrar un conjunto de rutas, cada una realizada por un solo vehículo que comienza y termina en un almacén, de manera que se satisfaga la demanda de todos los clientes, se cumplan las restricciones impuestas a la solución y se minimice el costo global de la transportación \cite{toth@vrp}.

% La aplicación del Problema de Enrutamiento de Vehículos a numerosas situaciones del mundo real ha demostrado que el uso de procedimientos computarizados para la planificación de la distribución de mercancías logra reducir sustancialmente el costo de la transportación \cite{rasku@automated, toth@vrp}. El impacto en el sistema económico global que producen estos ahorros es innegable debido a la ubicuidad del problema de transportación de mercancías en todas las etapas de producción en los sistemas de distribución.

% Los modelos matemáticos que describen las características de cada uno de estos problemas pueden ser un factor crítico en el rendimiento de las soluciones propuestas cuando se resuelven mediante métodos exactos. Sin embargo, los primeros modelos surgidos para describirlos han sido enriquecidos con características y restricciones que cubren un mayor número de situaciones que las descripciones iniciales de los VRP, convirtiendo el proceso de construcción de los modelos en una tarea complicada, con mayor consumo de tiempo y propensa a errores humanos.

% Las dificultades planteadas anteriormente serían resueltas si se contara con una herramienta que asista al usuario en la creación de modelos matemáticos, dadas las características y restricciones de un VRP. Un componente esencial de dicho sistema sería la verificación de la correctitud de un modelo para un VRP dado.

% Hasta donde se ha podido comprobar en las búsquedas realizadas, no se reporta en la literatura ninguna herramienta que verifique automáticamente si un modelo para un VRP específico está correcto o no. La mayoría de los trabajos que tratan sobre los problemas de enrutamiento de vehículos están enfocados en las soluciones y métodos de solución de los mismos, en lugar de sus modelaciones. Los modelos son generados manualmente o mediante un proceso de prueba y error llevado a cabo por humanos. De la literatura consultada, la investigación realizada por Lazaros Tatalopoulos \cite{tatalopoulos@thesis} es la que más similitudes guarda con este trabajo, pues implementa una herramienta para la modelación de diferentes VRP, permitiéndole al usuario la selección de las características que se adaptan a su problema para ofrecer como salida el modelo correspondiente. Además, el modelo resultante se encuentra predefinido en la herramienta, mientras que el sistema que se propone en este trabajo puede extenderse para validar cualquier variante del VRP.

% El objetivo de este trabajo es implementar una herramienta para determinar si un modelo dado para una variante específica del VRP es correcta o no. Para ello se diseñará un mecanismo de validación basado en unidades de prueba. Los objetivos específicos planteados para dar cumplimiento al objetivo general son:

% \begin{itemize}
% 	\item Caracterizar las distintas versiones de VRP más conocidas, así como los modelos propuestos para los mismos.
% 	\item Proponer un mecanismo para describir las características iniciales de un problema de VRP.
% 	\item Estudiar el uso de unidades de pruebas para detectar errores y fallas en modelos propuestos.
% 	\item Diseñar un mecanismo que permita generar unidades de prueba para comprobar si un modelo refleja correctamente todas las características del VRP descrito.
% 	\item Diseñar un sistema de generación de soluciones factibles e infactibles que respondan a las características del problema.
% \end{itemize}

% Las unidades de prueba son un mecanismo utilizado en ingeniería de software para asegurar que un fragmento de código se comporta de la manera deseada. La idea es implementar un método o función que llame a otro método y que compruebe posteriormente su funcionamiento a partir de los resultados que ofrece o los estados que modifica. En la literatura existen numerosas técnicas para elaborar un sistema de unidades de prueba correcto. En este trabajo se usarán algunas de estas técnicas para comprobar la correctitud de un modelo matemático en lugar de un fragmento de código.

% El uso de la estrategia de unidades de pruebas en este trabajo permite asegurar la verificación de un grupo representativo de situaciones mediante la generación de soluciones que describan cada una de las características del problema, ya sea porque las satisfacen o porque incumplen un subconjunto de ellas.

% Para determinar la correctitud de un modelo se crea un conjunto de soluciones y problemas donde, para cada par solución-problema se conoce si la solución es factible o no para el problema generado. Al crear soluciones infactibles se especifican las características que deben incumplir. A partir de las soluciones creadas se obtienen las variables y parámetros del modelo que describen esas mismas soluciones. Al mismo tiempo, para cada restricción del modelo se crean funciones que reciben como entrada las variables del modelo y verifican si la solución satisface esa restricción o no.

% Con las soluciones y funciones generadas, se verifica si el modelo parece estar correcto porque las condiciones impuestas a cada solución fueron cumplidas en todas las pruebas, o está definitivamente incorrecto porque una solución factible no cumplió al menos una restricción o una infactible satisfizo todas las restricciones.

% Este documento está estructurado en cuatro capítulos. En el capítulo \ref{chap:preliminaries} se define el marco teórico del trabajo, que abarcará los problemas de enrutamiento de vehículos, los modelos de optimización matemática, las unidades de prueba y el lenguaje Common Lisp. En el capítulo \ref{chap:system} se presenta el funcionamiento general del sistema, mientras que en el capítulo \ref{chap:implementation} se describen los detalles de implementación y los recursos para extender el sistema. En el capítulo \ref{chap:results} se muestra la validación de un modelo para el CVRP. Finalmente, se muestran las conclusiones y, como parte de ellas, las recomendaciones del trabajo.

