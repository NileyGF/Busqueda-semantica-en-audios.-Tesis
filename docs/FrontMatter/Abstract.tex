\chapter*{Resumen} 
El lenguaje natural, como una de las interfaces más intuitivas conocidas por los humanos, tiene un potencial significativo 
como intermediario de muchas tareas que involucran la interacción humano-computadora, especialmente en campos enfocados en 
aplicaciones como la Recuperación de Información Musical (MIR, por sus siglas en inglés). El objetivo de este estudio fue 
investigar y diseñar un prototipo para la recuperación de música utilizando texto. Esto implicó recuperar contenido musical 
de un conjunto de candidatos que mejor coincidiera con una descripción dada en lenguaje natural, una tarea que ha recibido atención 
limitada en la literatura existente.\\
Este trabajo exploró un enfoque basado en modelos de MIR para extraer características de bajo y alto nivel de un archivo de música. 
Estas características se utilizaron para generar descripciones en lenguaje natural. Para recuperar, se utilizaron \textit{embeddings} de BERT 
(\textit{Bidirectional Encoder Representations from Transformers}) para calcular la similitud semántica entre la consulta del usuario 
y las descripciones de la base de datos. El resultado produjo un prototipo prometedor, estableciendo una base sólida para futuros esfuerzos 
en esta línea de investigación.\\

{\bf Palabras claves: recuperación de información musical, recuperación basada en texto, descripción de música, vectores de \textit{embeddings}, Sentence BERT} 

\chapter*{Abstract}
Natural language, as one of the most intuitive interfaces known to humans, holds significant potential in mediating many 
tasks involving human-computer interaction, particularly in application-focused fields like Music Information Retrieval 
(MIR). The aim of this study was to research and design a prototype for text-to-music retrieval. This involved retrieving 
music content from a pool of candidates that best matched a given natural language description, a task that has received 
limited attention in existing literature. \\
This work explored an approach that relied on MIR models to extract low-level and high-level features from a music file. 
These features were employed to generate natural language descriptions. For retrieval purposes, BERT embeddings were 
used to calculate semantic similarity between the user query and the database descriptions. The outcome yielded a promising 
prototype, laying a strong foundation for future efforts in this line of research.\\

{\bf Keywords: music information retrieval, text-based retrieval, music captioning, embeddings vectors, Sentence BERT} 