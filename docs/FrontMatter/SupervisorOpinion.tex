\chapter*{Opinión del tutor}
\small
La estudiante Niley González Ferrales desarrolló satisfactoriamente el trabajo de diploma titulado “Recuperación semántica de música utilizando embeddings y modelos de clasificación”. En este trabajo la estudiante propuso el diseño e implementación de un prototipo de una plataforma que permita realizar consultas en lenguaje natural sobre una base de datos de música y obtener resultados con significación semántica.

El trabajo propone un acercamiento a la recuperación de música con consultas en lenguaje natural, que aprovecha el poder de los clasificadores de MIR y los embeddings de oraciones. Es un trabajo inciático que explora el potencial de utilizar clasificadores de aprendizaje automático para extraer características matizadas de la música, en un intento de traducir sus cualidades auditivas en una descripción en lenguaje natural. Es un primer paso en tratar de establecer un puente entre la naturaleza abstracta de la música y el ámbito lingüístico de la comunicación humana, disminuyendo la brecha semántica y despejando el camino para mejores sistemas de recuperación de música. Esta investigación abre una nueva línea de investigación dentro del Grupo de Investigación en Inteligencia Artificial.

Para poder afrontar el trabajo, la estudiante tuvo que revisar literatura científica relacionada con la temática así como soluciones existentes y bibliotecas de software que pueden ser apropiadas para su utilización. Todo ello con sentido crítico, determinando las mejores aproximaciones y también las dificultades que presentan.

Todo el trabajo fue realizado por la estudiante con una elevada constancia, capacidad de trabajo y habilidades, tanto de gestión, como de desarrollo y de investigación. 

Por estas razones pedimos que le sea otorgada a la estudiante Niley González Ferrales la máxima calificación y, de esta manera, pueda obtener el título de Licenciada en Ciencia de la Computación.

 \begin{flushright}
   \underline{\hspace{6.5cm}}\\
   Dr. Yudivián Almeida Cruz
   
   Facultad de Matemática y Computación
   
   Universidad de la Habana
   
   Enero, 2024
 \end{flushright}

\normalsize