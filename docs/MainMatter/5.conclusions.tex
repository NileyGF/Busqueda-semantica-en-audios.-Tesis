%===================================================================================
% Chapter: Conclusions
%===================================================================================
\chapter*{Conclusiones}
\label{chap:conclusions}
\addcontentsline{toc}{chapter}{Conclusiones}
%===================================================================================
Esta tesis presentó un enfoque para la recuperación de música con consultas en lenguaje natural, que aprovecha el poder de los clasificadores de MIR y los \textit{embeddings} de oraciones.

A través de la exploración, se ha mostrado el potencial de utilizar clasificadores de aprendizaje automático (\textit{machine learning}) para extraer características matizadas de la música, en un intento de traducir sus cualidades auditivas en una descripción en lenguaje natural. Este paso representa un puente crítico entre la naturaleza abstracta de la música y el ámbito lingüístico de la comunicación humana, reduciendo la brecha semántica y allanando el camino para mejores sistemas de recuperación de música. Ahí radica la importancia de continuar por esta vertiente de investigación y encontrar otras formas de realizar la transformación de las características musicales a textos de que representen la percepción humana. Es probable que una forma de mejorar en este esfuerzo sea siguiendo los grandes avances en modelos de lenguaje de los últimos años (ya sea TableGPT u otro enfoque).

%Nuestra investigación resalta el potencial de aplicaciones prácticas dentro de la industria musical, sugiriendo un enfoque que podría revolucionar la forma en que los usuarios interactúan y descubren la música. Desde recomendaciones personalizadas hasta interfaces de búsqueda más intuitivas, el impacto de esta investigación se extiende más allá del ámbito académico, resonando profundamente dentro del campo de la tecnología musical.

Es importante reconocer las limitaciones del trabajo. Aunque el prototipo es prometedor, todavía existen desafíos por abordar, como la interpretabilidad de las descripciones en lenguaje natural y la escalabilidad del proceso de recuperación. También es importante señalar que aún no existen datasets para la tarea de recuperación de texto a música, y los datasets adaptables para la tarea no son de un tamaño adecuado. 
Se reconoce que las causas de la indisponibilidad de datasets también deben abordarse. Entre ellas se encuentran: el problema mencionado de los derechos de autor (\textit{copyright}), el hecho de que la alineación de texto y audio es una tarea computacional difícil que debe ser supervisada por humanos, y la ausencia de un estándar ampliamente adoptado para la representación multimodal de la música. Enfoques recientes de aprendizaje de representaciones multimodales han mostrado avances en muchos dominios al aprovechar enormes datos de la web, pero no en dominios musicales (al menos no en investigaciones públicas).

El potencial para futuros avances en la recuperación de música y sistemas de recomendación es vasto. Se espera que esta investigación siente las bases para estos esfuerzos futuros. Este es un primer intento de construir una interfaz de lenguaje natural de forma libre para música, y hay mucho espacio para mejorar.

%===================================================================================
% Chapter: Recommendations
%===================================================================================
\chapter*{Recomendaciones}
\label{chap:recommendations}
\addcontentsline{toc}{chapter}{Recomendaciones}
%===================================================================================

Para futuras investigaciones se recomienda utilizar TableGPT \cite{Gong2020TableGPTFT} u otro modelo de \textit{table-to-text generation} para la tarea de obtener una descripción de la música a partir de una lista de features y comparar el resultado con los presentados en este trabajo, que utilizan un acercamiento de fuerza bruta. \\
Otra idea a incorporar, que debería resultar en un prototipo más eficiente, es aumentar la cantidad y diversidad de modelos de clasificación en el sistema de extracción de features (sec. \ref{subsec:essentia}).

En la sección de discusión de los resultados de la experimentación (sec. \ref{sec:final-discution}), se plantea que \textit{ext\_corpus} no es una alternativa factible para mejorar las descripcioes; sin embargo se podría evaluar un corpus conformado por un par de vectores de \textit{embeddings} por cada canción: uno de descripciones en forma de texto y otro formado por tags (o sea, una combinación de las ideas del primer y el tercer corpus propuesto). Además, en el experimento 1 (sec. \ref{sec:experiment1}) se analiza que los resultados mejoran cuando las consultas tienen la misma forma que las descripciones. Un corpus mixto puede ser una solución robusta al hecho de que no se puede preveer que tipo de consulta utilizarán los usuarios en la búsqueda.

Para finalizar se recomienda evaluar el prototipo realizando estudios de ablación para comprobar el impacto de cada parte del sistema. Por ejemplo, cuánto mejoran los resultados al incluir la búsqueda utilizando \textit{embeddings} de BERT en comparación con el modelo vectorial (de bolsa de palabras) tradicional.
