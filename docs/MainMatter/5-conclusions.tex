\chapter*{Conclusiones}\label{chap:conclusions}
\addcontentsline{toc}{chapter}{Conclusiones}

% La validación de modelos de optimización matemáticos es una tarea que se realiza manualmente y requiere de la inversión de una gran cantidad de tiempo. En este trabajo se propuso una herramienta que permite realizar esta actividad de manera automática cuando el problema que se quiere modelar es un VRP.

% El sistema de validación se basa en la estrategia de unidades de prueba. En el caso de esta investigación, en lugar de verificar la correctitud de un fragmento de código, se analiza la correctitud de un modelo de optimización matemática para un VRP específico. La idea es generar soluciones factibles e infactibles atendiendo a un conjunto de requerimientos y después evaluarlas en las restricciones del modelo. Si la solución es factible, entonces todas las restricciones deben ser cumplidas. Si la solución es infactible, entonces al menos una restricción debería fallar. Si estas condiciones se cumplen, entonces el modelo parece estar correcto; de lo contrario, se puede afirmar que está incorrecto, y se presentan ejemplos que así lo demuestran.

% Para el sistema de validación se concibieron cuatro componentes. El primero es un generador de soluciones de alto nivel que construye soluciones factibles e infactibles, e instancias del problema a partir de un requerimiento. La salida de este sistema se usa como entrada del segundo componente, el generador de soluciones de bajo nivel, que devuelve los valores correctos para las variables, parámetros y conjuntos del modelo. Para poder evaluar las soluciones, las restricciones del modelo se transforman en funciones. Este proceso ocurre en el tercer componente llamado generador de funciones. El último componente es el evaluador del modelo cuya salida permite saber si el modelo está correcto o no.

% El sistema implementado, no solo cumple con el objetivo propuesto de validar automáticamente un modelo para un VRP específico, sino que brinda las herramientas necesarias para realizar un análisis posterior en caso de fallo. Cuando un modelo se determina como incorrecto, se proveen las soluciones que no pasaron las pruebas y se muestra la salida que obtuvieron cada una al evaluarse en las restricciones. Esto permite encontrar los fallos del modelo y proponer una solución con mayor rapidez.

% Actualmente, el sistema implementado solo permite validar el modelo de flujo por mercancías del CVRP. Sin embargo, cuenta con los recursos necesarios para incorporar nuevos modelos y problemas. Por esta razón, se propone como recomendación extender el sistema para validar nuevos modelos. Además, se considera interesante, una vez determinado que un modelo está incorrecto, encontrar la o las restricciones erróneas.
